卒業論文要旨 - 2019年度 (令和元年度)
\begin{center}
\begin{large}
\begin{tabular}{|M{0.97\linewidth}|}
    \hline
      \title \\
    \hline
\end{tabular}
\end{large}
\end{center}

~ \\

従来のオーディオ伝送の諸問題を解決するために、IPオーディオ伝送が登場した。のちに乱立するさまざまなIPオーディオ伝送規格を標準化するAES67規格が策定された。ブラックボックスとなっている商用規格を使わずとも、自由に機器やアプリケーションを開発することができるようになった。

そこで、AES67に準拠したアプリケーションを実装し、実際の業務用オーディオ現場における利用可能性を検証することにした。

業務用オーディオシステムに必要な要件は、遅延の少なさ、高音質、伝送の安定性が挙げられる。

本論文では、AES67のソフトウェアオーディオ伝送は、業務用オーディオにおける要件を満たすと考えられる。

~ \\
キーワード:\\
\underline{1. AES67},
\underline{2. Audio over IP}
\begin{flushright}
\dept \\
\author
\end{flushright}
