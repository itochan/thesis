卒業論文要旨 - 2019年度 (令和元年度)
\begin{center}
\begin{large}
\begin{tabular}{|M{0.97\linewidth}|}
    \hline
      \title \\
    \hline
\end{tabular}
\end{large}
\end{center}

~ \\

従来のオーディオ伝送の諸問題を解決するために、IPオーディオ伝送が登場した。のちに乱立するさまざまなIPオーディオ伝送規格を標準化するAES67規格が策定された。ブラックボックスとなっている商用規格を使わずとも、自由に機器やアプリケーションを開発することができるようになった。

業務用オーディオシステムに必要な要件は、遅延の少なさ、高音質、伝送の安定性が挙げられる。

遅延を感じ、バイオリンの演奏に支障を与える程度を計測する実験で、10ミリ秒であればほとんどの被験者が遅延を感じないという研究がある。音が入力されてからスピーカーで出力されるまでの遅延時間には、伝送経路以外の時間も含まれる。例えば、ワイヤレスマイクを使用した場合や、エフェクタ等の信号処理で遅延が発生する。よって、オーディオ伝送において遅延を少なくすることが重要だ。DanteとよばれるIPオーディオ伝送規格では、デフォルトで1ミリ秒の遅延を許容する設定になっている。

デジタルオーディオ伝送は、伝送経路において信号の劣化しないため業務用途で頻繁に使われているが、それらは非圧縮で伝送されている。不可逆圧縮する場合、圧縮と展開を繰り返すたびに、劣化してしまう。よって、複雑な信号処理が行われることもある業務用オーディオでは非圧縮伝送は必須条件だ。PCM 48kHz/24bit程度であれば望ましい。

伝送経路の安定性も業務用途によって欠かせない要素である。プロのアーティストによるライブやコンサートでは、一瞬たりとも音が途切れることは許されない。機材トラブルやケーブルの断線が起きれば、オーディオ伝送は行われなくなる。デジタルオーディオ伝送規格では、伝送経路の冗長化をサポートしているものがある。

これらの業務用オーディオシステムに必要な要件を踏まえ、AES67によるソフトウェア実装の実際の業務用オーディオ現場における利用可能性を検証することにした。実装においてGo言語を採用した。移植性の高さと高速に動作するためである。

その結果、遅延量は想定する1ミリ秒を下回り、AES67のソフトウェアオーディオ伝送は、業務用オーディオにおける要件を満たすと考えられる。

~ \\
キーワード:\\
\underline{1. AES67},
\underline{2. Audio over IP}
\begin{flushright}
\dept \\
\author
\end{flushright}
