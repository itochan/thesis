\chapter{設計}
\label{chap:design}

本章では、AES67を用いたIPベースのオーディオ伝送をするアプリケーションの設計について述べる。

\section{アプリケーションの構成}

本研究で実験を行うために必要な、アプリケーションの機能は大きく分けて2つに分類できる。AES67オーディオストリームの送信と、受信を行うものだ。

\section{AES67ストリーム送信機能}

AES67オーデイオストリームの送信機能について述べる。

\subsection{IPマルチキャスト送信}

はじめに、AES67ストリーム送信機能のうち、IPマルチキャストによって送信する機能について述べる。

IPv4マルチキャストアドレスはRFC5771\cite{rfc5771}において、224.0.0.0/4のアドレス空間を用いると定められている。Danteの仕様では239.69.0.0/16のアドレスを使うと定められているため、本実装ではそのアドレス範囲を使うものとする。

\subsection{IPユニキャスト送信}

次に、AES67ストリーム送信機能のうち、IPユニキャストによって送信する機能について述べる。

\section{AES67ストリーム受信機能}

AES67オーデイオストリームの受信機能について記す。

\section{まとめ}

次章では、本章で設計したアプリケーションの実装を行う。ところで、設計と実装は同じ章でもよいのではないだろうか。
