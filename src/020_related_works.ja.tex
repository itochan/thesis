\chapter{関連技術}
\label{chap:related_works}

本章では,本研究と同様のIPを用いてオーディオを伝送する手法について,関連技術を示す.

\section{AES/EBU}

AES3\cite{aes3-1992}とEBU Tech.3250-E\cite{ebutech-3250-e}は,1985年に策定されたデジタルオーディオ伝送規格である.オーディオ専門家の国際組織である米国オーディオ技術者協会(AES)\footnote{Audio Engineering Society, Inc. \url{http://www.aes.org/}}と,欧州放送連合(EBU)\footnote{European Broadcasting Union \url{https://www.ebu.ch/home}}の両者によって策定された規格だ.なお,両者の名前をとって一般的にAES/EBUの名称で呼ばれている.

また,AES3はIEC(International Electrotechnical Commission)によって標準化もされている.IECによる標準化では4つのパートで構成されており,IEC 60958-3では民生用途向け(いわゆるS/PDIF)と,IEC 60958-4では業務用途向けのものが定められている.

接続するインターフェイスはIEC規格によって策定されている.業務用のものでは,アナログオーディオ伝送にも用いられるXLRコネクタによるバランス(平衡)接続や,BNCコネクタを用いた同軸ケーブルによるアンバランス(不平衡)接続,さらにはD-subコネクタを用いた音響機器メーカーによる独自規格が存在する.

\subsection{AES-3id-1995}

BNCコネクタを用いたアンバランス接続は,AES3-id-1995という名称で策定されている.AES/EBUでは,インピーダンスが110Ωのケーブルを用いる.バランス接続は,ノイズの影響を受けにくく安定した伝送が期待できるものの,アンバランス接続の同軸ケーブルに比べ高価である.BNCコネクタが付いたインピーダンスが75Ωの同軸ケーブルは,業務用映像伝送の分野で主流だ.大規模なホールのような会場では予めさまざまな場所に配線しておき,パッチ盤で接続するという手法がある.パッチ盤のコネクタがBNCであれば何を伝送してもよいため,AES/EBUで同軸ケーブルを用いて伝送するというアイデアは,映像技術の団体米国映画テレビ技術者協会(SMPTE)\footnote{Society of Motion Picture and Television Engineers \url{https://www.smpte.org/}}からAESに対して要望が出ていたという経緯がある\cite{aes3id-1995-column}.同軸ケーブルを用いた伝送は,信号を補償(ブースト)することでより長距離伝送が可能だ.

\section{S/PDIF}

前述したとおり業務用のAES/EBUから派生したの民生用向けの規格は,IEC 60958-3によって標準化されている.この規格は,一般的にS/PDIF(Sony/Philips Digital Interface)と呼ばれる.日本のソニーとオランダのフィリップスが開発したためそのように呼ばれている.

AES/EBUでは,民生用では馴染みの薄いXLRコネクタやBNCコネクタが用いられてきたが,S/PDIFでは光デジタル音声端子(Optical)と同軸デジタル音声端子(Coaxial)の2種類が存在する.光デジタル端子は,東芝が提唱したTOSLINKと呼ばれる角型コネクタ,Mini-TOSLINKと呼ばれる3.5mmステレオミニプラグの形状をしたコネクタがある.同軸端子は,RCAコネクタが使われている.

Mini-TOSLINKは,以前AppleのパソコンであるiMacやMac mini,MacBook Proにも搭載されていた\footnote{MacBook Proは2015年発売の機種まで,3.5mmステレオミニジャックにアナログ入力とコンボジャックで搭載していた. \url{https://support.apple.com/kb/SP719?locale=ja_JP&viewlocale=ja_JP}}.

\subsection{AES/EBUとS/PDIFの違い}

AES/EBUは業務用でS/PDIFは民生用だが,どのような違いがあるだろうか.以下に比較表を記した.業務用オーディオは,可能な限り伝送距離を長くし,XLRコネクタで接続した場合バランス接続による安定性をとっている.一方,民生用オーディオは出力レベルを低くし,伝送距離も10m程度である.ただし,著作権保護信号を伝送することができ,受信機が信号を認識するとその機器で録音ができなくなるようだ.

\begin{table}[htb]
  \begin{center}
    \caption{AES/EBUとS/PDIFの比較\cite{aesebuandspdif}}
    \begin{tabular}{c|ccc} \hline
      & AES3-1992 (r1997) & AES-3id-1995\footnote{SMPTE 276M-1995} & S/PDIF\footnote{IEC 60958-3} \\ \hline \hline
      インターフェイス & バランス & アンバランス & アンバランス \\
      コネクタ & 3ピンXLR & BNC & RCA / TOSLINK \\
      信号レベル & 2--7V peak to peak & 1.0--1.2V peak to peak & 0.5--0.6V peak to peak \\
      最小信号レベル & 0.2V & 0.32V & 0.2V \\
      最大伝送距離 & 100m & 1000m & 10m \\
      コピー制御 & 不可 & 不可 & 可能 \\
      ビット深度 & 24bit & 24bit & 20bit(24bitに拡張可能)\footnote{4bit分予備領域として確保されており,20+4bitを合わせて送信できる.} \\ \hline
    \end{tabular}
  \end{center}
\end{table}

\section{SDI}

\section{Dante}

\section{RAVENNA}

\section{AES67}
