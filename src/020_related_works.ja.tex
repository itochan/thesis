\chapter{関連技術}
\label{chap:related_works}

本章では,本研究と同様のIPを用いてオーディオを伝送する手法について,関連技術を示す.

\section{AES/EBU}

AES3\cite{aes3-1992}とEBU Tech.3250-E\cite{ebutech-3250-e}は,1985年に策定されたデジタルオーディオ伝送規格である.オーディオ専門家の国際組織である米国オーディオ技術者協会(AES)\footnote{Audio Engineering Society, Inc. \url{http://www.aes.org/}}と,欧州放送連合(EBU)\footnote{European Broadcasting Union \url{https://www.ebu.ch/home}}の両者によって策定された規格だ.なお,両者の名前をとって一般的にAES/EBUの名称で呼ばれている.

また,AES3はIEC(International Electrotechnical Commission)によって標準化もされている.IECによる標準化では4つのパートで構成されており,IEC 60958-3では民生用途向け(いわゆるS/PDIF)と,IEC 60958-4では業務用途向けのものが定められている.

接続するインターフェイスはIEC規格によって策定されている.業務用のものでは,アナログオーディオ伝送にも用いられるXLRコネクタによるバランス(平衡)接続や,BNCコネクタを用いた同軸ケーブルによるアンバランス(不平衡)接続,さらにはD-subコネクタを用いた音響機器メーカーによる独自規格が存在する.

\subsection{AES-3id-1995}

BNCコネクタを用いたアンバランス接続は,AES3-id-1995という名称で策定されている.AES/EBUでは,インピーダンスが110Ωのケーブルを用いる.バランス接続は,ノイズの影響を受けにくく安定した伝送が期待できるものの,アンバランス接続の同軸ケーブルに比べ高価である.BNCコネクタが付いたインピーダンスが75Ωの同軸ケーブルは,業務用映像伝送の分野で主流だ.大規模なホールのような会場では予めさまざまな場所に配線しておき,パッチ盤で接続するという手法がある.パッチ盤のコネクタがBNCであれば何を伝送してもよいため,AES/EBUで同軸ケーブルを用いて伝送するというアイデアは,映像技術の団体米国映画テレビ技術者協会(SMPTE)\footnote{Society of Motion Picture and Television Engineers \url{https://www.smpte.org/}}からAESに対して要望が出ていたという経緯がある\cite{aes3id-1995-column}.同軸ケーブルを用いた伝送は,信号を補償(ブースト)することでより長距離伝送が可能だ.

\section{S/PDIF}

\section{SDI}

\section{Dante}

\section{RAVENNA}

\section{AES67}
