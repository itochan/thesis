\chapter*{謝辞}\markboth{謝辞}{謝辞}
\addcontentsline{toc}{chapter}{謝辞}
\label{thanks}

本研究を進めるにあたり、ご指導いただいた慶應義塾大学環境情報学部教授 村井純博士、同学部教授 中村修博士、同学部教授 楠本博之博士、同学部教授 高汐一紀博士、同学部教授 Rodney D.Van Meter III博士、同学部准教授 植原啓介博士、同学部教授 三次仁博士、同学部教授 中澤仁博士、同学部教授 武田圭史博士、同大学政策・メディア研究科特任准教授 佐藤雅明博士、同大学政策・メディア研究科特任教授 鈴木茂哉博士、同大学SFC研究所上席所員 斉藤賢爾博士に感謝いたします。Arch研究グループでは、日頃より研究をご指導いただきました松谷健史博士、空閑洋平博士、大江将史博士に感謝いたします。特に、村井純博士は研究活動以外においても未熟な私に人生や恋愛とは何かということを教えていただきました。

本研究のきっかけとなったヤマハ株式会社 深沢豪氏、山根章生氏からは製品を開発する立場からのご意見をいただくことができ、大いに助けられました。重ねて感謝申し上げます。

プロフェッショナルオーディオへの興味を持つきっかけとなった、福利厚生団体音像工房での活動で知り合った金子浩幸氏は私に多くの音響や映像の仕事を紹介していただきました。また、大学のイベントで機材の特別貸出を実施していただいた、慶應義塾大学 湘南藤沢メディアセンターおよび同メディアセンター マルチメディアサービス担当 長坂功氏には深く感謝申し上げます。

村井・楠本・中村・高汐・バンミーター・植原・三次・中澤・武田合同研究プロジェクトに所属している学部生、大学院生、卒業生の皆さまに感謝いたします。長い研究室生活で多くの時間を過ごした、菅藤佑太氏、安井瑛男氏、石川達敬氏、阿部涼介氏、豊田安信氏、Korry Luke氏、小西遼氏、矢内洋祐氏、用澤玄汰氏、真島大樹氏、山田真也氏、橘直雪氏、水野史暁氏、平野孝徳氏、森島隆成氏、米山涼氏、上田侑真氏、坂本優
太氏、根本樹氏、さらには1年休学したため一緒に卒論を戦った栗原祐二氏、深川祐太氏、鈴木雄祐氏、島津翔太氏、井手田悠希氏、勝又海氏、西田亘氏、山田航太郎氏に感謝いたします。楽しいことも苦しいことも、どんな出来事でも包み隠さずに話したり相談できる大切な仲間です。また、研究室を日々きれいに使うとともに居心地をよくしてくれた学部生と大学院生の皆さまには大変感謝しています。本論文執筆に際して \LaTeX テンプレートを作成し研究室内で知見を共有してくれた、研究室の同期と先輩方に感謝申し上げます。

研究室に所属している間はずっとArch研究グループにいましたが、ICAR研究グループの皆さんと佐藤雅明博士とは合宿を通じて夜通し語り合ったり、ICARに所属していないながらも仲良くしていただきました。

本研究は慶應義塾大学環境情報学部、つまりSFC(湘南藤沢キャンパス)に入学していなければ、おそらく行うことはなかったのではないかと思います。入学以前から知り合いでSFCに通っていた、山中勇成氏、黒米祐馬氏、小島凌汰氏がいたおかげでSFCに進学することを考えるようになりました。AO(アドミッションズ・オフィス)入試を受けるにあたり、当時通っていた東京都立一橋高等学校で3年次担任の榎本津加子氏、高校から大学までの5年間アルバイトをしていたピクシブ株式会社の小芝敏明氏には私の推薦文を書いていただくなど、多大なるご支援をいただきました。お二方がいなければ、今のSFCでの生活と研究活動はありませんでした。深く感謝いたします。

中学卒業後、最初に入った高校は通信制の高校でした。1年通って転学することになってしまいましたが、SFCにいる今の私がいるのはそのとき転学を決断していたのも関係しています。転学の前に通っていた科学技術学園高等学校で担任だった金子知史氏には、転学するにも関わらず相談やその後押しをしてくださり、感謝しています。

また、入学に関して欠かすことのできなかったものが奨学金です。奨学金なくして大学に進学することは不可能でした。高校で推薦をいただき、公益財団法人日本教育公務員弘済会 東京支部から20万円、公益信託中村奨学基金から18万円(いずれも給付奨学金)のご支援をいただきました。さらに、独立行政法人日本学生支援機構からは入学時特別増額貸与奨学金50万円と、第二種奨学金毎月12万円(機関保証)48ヶ月の貸与奨学金を利用しています。奨学金受け取りに関して、推薦をしていただいた東京都立一橋高等学校教諭の皆さま、公益信託中村奨学基金の審査を行った三菱UFJ信託銀行 リテール受託業務部公益信託課の皆さま、日本学生支援機構 入学時特別増額貸与奨学金のつなぎ融資をしていただいた中央労働金庫 新宿支店の皆さまにはお礼申し上げます。

私がSFCに通い続けた、学部の4年間と1年間の休学で得られた体験は一生の宝物です。印象深い授業や、研究室やサークルを通して出会えた友人からは多くの刺激を受けるとともに深く考えさせられました。SFCで記憶に残る授業「データ・ドリブン社会の創発と戦略」「データ・ドリブン社会の創発と戦略(応用)」を通じて知り合うことになった、慶應義塾大学環境情報学部教授 安宅和人博士からは、激動の時代を生き抜く知恵と勇気をいただくことができました。私が所属していたサークル・団体、福利厚生団体音像工房、SFC CLIP、アカペラシンガーズK.O.E.の皆さまには大変お世話になりました。

1年次の必修授業「情報基礎」でSA(スチューデント・アシスタント)をしていたのもいい経験でした。3年間・6学期と計8クラスも受け持つことになりました。授業の担当講師であった独立行政法人産業技術総合研究所 橋本尚久博士、慶應義塾大学理工学部 栗原聡博士、同大学政策・メディア研究科後期博士課程 阿部涼介氏は、SAとして私を迎え入れてくれるのみならず、さまざまな相談にも乗っていただきました。一緒にSA/TAをしていた、尾崎周也氏、高塚大暉氏、深川祐太氏、山口航平氏、加藤太一氏、大塚崇夫氏、鈴木雄祐氏、栗原祐二氏、種谷望氏と教えた日々もいい思い出です。ここで出会えた履修者も、キャンパスで出会ったときに声をかけてくれるなど、教えることができてよかったと思う次第です。

そして、語らずにはいられない何よりも忘れられないことがあります。最後まで成し遂げられなかったSFCでしか、私にしか味わうことのできない恋愛の思い出です。これから先の人生でどんなことがあっても、私の心には残り続けることでしょう。ここに、好きな人の名前を書ければかっこよかったのかもしれません。しかし残念ながら、それが叶うことはありませんでした。

これまでの人生で、何度か恋をすることがありました。恋をする、人を好きになるということは、好きでいる間は幸せなものの相手が好きとも限らなかったり、時には悲しませてしまうこともありました。この経験を忘れることなく、これから生きていくうえでの指針にして大切にしていこうと思います。SFCではいい出会いをすることができました。ありがとうございました。好きな人の幸せは、私の幸せです。私も負けないように幸せになろうと思います。そうなったらそれぞれにお互いの大切な人がいるかと思いますが、いつかまた笑顔で会える日を楽しみにしています。

恋人のいない私に構ってくれたSFCの皆さん(自分の恋人がいるのに…)と、寂しいときに話し相手になってくれた家族である父 尚生、母 真由美、妹 さくら、猫 ささみに感謝しています。

これで最後です。すべての人の名前をここで挙げることはできませんが、SFCで出会うことができたすべての人に感謝します。SFCで出会えたみんなが大好きです。本当にありがとうございました。
