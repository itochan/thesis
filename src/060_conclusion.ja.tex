\chapter{結論}
\label{chap:conclusion}

\section{本研究のまとめ}
\label{section:conclusion}

本研究ではAES67のソフトウェア実装が業務用オーディオシステムにおいて適用可能か検証するため、AES67オーディオストリームの送受信アプリケーションを実装し、評価を行った。

\section{本研究の結論}

\ref{chap:evaluation}章で行った評価結果から、AES67のソフトウェアオーディオ伝送は、\ref{sec:pro_audio_requirement}節で述べた業務用オーディオにおける要件を満たすと考えられる。

\section{今後の課題}

\subsection{RTPパケット送信間隔の適正値}

\ref{sec:evaluation_latency}節で、1ミリ秒間隔で1チャンネルあたり48サンプルずつ送信した結果遅延が発生したことを述べた。

1ミリ秒のタイマーを設定し、1ミリ秒間隔で送信できないのはコンピュータのタイマ性能に起因するものではないかと考えられる。したがって、5ミリ秒間隔で240サンプルずつ、10ミリ秒間隔で480サンプルずつ、といったように同時に送信するサンプル数を増やすことで安定した間隔で行えるのではないだろうか。

\subsection{伝送経路の冗長化}

業務用オーディオでは、音を高音質で伝送するとともに、確実に伝送することが必要である。SMPTEは、ST2110\cite{smpte-st2110}というオーディオとビデオをまとめて送る規格を策定した。その中に、冗長化に関する仕様が定められている。冗長化を行うことで、より業務用途に適したIPオーディオ伝送が実現するだろう。

\subsection{ハードウェア実装}

AES67の伝送を、ハードウェアで実装することでより高速に伝送できるのではないかと考えている。Field Programmable Gate Array (FPGA)で実装することを今後検討したい。
