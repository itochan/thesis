\chapter{結論}
\label{chap:conclusion}

\section{本研究のまとめ}
\label{section:conclusion}

オレオレ実装でXXXmsの遅延で送れてしまった。

\section{本研究の結論}

これらの結果により、雑な実装でも案外送れちゃうことがわかった。

ただし、雑なベンチマークにより外部要因を排除できていない。Linuxカーネルのパラメータをいじる必要がありそうだ。

\section{今後の課題}

\subsection{RTPパケット送信間隔の適正値}

\ref{sec:evaluation_latency}項で、1ミリ秒間隔で1チャンネルあたり48サンプルずつ送信した結果遅延が発生したことを述べた。

1ミリ秒のタイマーを設定し、1ミリ秒間隔で送信できないのはコンピュータのタイマ性能に起因するものではないかと考えられる。したがって、5ミリ秒間隔で240サンプルずつ、10ミリ秒間隔で480サンプルずつ、といったように同時に送信するサンプル数を増やすことで安定した間隔で行えるのではないだろうか。

\section{あとで整理する}

ハードウェア実装をするともっと早くなるからやってみたい。

SMPTE ST2110を参考にした冗長化。
