\chapter{評価}
\label{chap:evaluation}

本章では、本研究における評価手法とその結果について述べる。

\section{評価手法}

\ref{chap:implementation}章で実装したAES67送受信アプリケーションを使用する。2台のPCでそれぞれ送信機能と受信機能を動作させ、伝送にかかる遅延秒数を計測する。

\section{評価環境}

次の構成のコンピュータを2台、ネットワークスイッチを介して接続する。

今回使用するネットワークスイッチは、Dante最適設定が搭載されている。Dante最適設定は、有効にした状態で計測を行った。

\begin{table}[htb]
  \label{tab:evaluation_computer}
  \caption{コンピュータ評価環境}
  \centering
  \begin{tabular}{c|l} \hline
    OS & Ubuntu 18.04 LTS \\ \hline
    CPU & Intel Core i7-9700K 3.60GHz \\ \hline
    RAM & 32GB \\ \hline
    SSD & Samsung SSD 970 EVO Plus 250GB \\ \hline
    NIC & StarTech ST1000SPEXI (Intel I210-AT) \\ \hline
  \end{tabular}
\end{table}

\begin{table}[htb]
  \label{tab:evaluation_network}
  \caption{ネットワーク評価環境}
  \centering
  \begin{tabular}{c|l} \hline
    ネットワークスイッチ & YAMAHA SWR2311P-10G \\ \hline
  \end{tabular}
\end{table}

\section{AES67伝送の結果}

\subsection{UDPパケット転送遅延}

最小91.125マイクロ秒、最大3490.794マイクロ秒、平均303.402マイクロ秒、中央値306.710マイクロ秒であった。

\subsection{パケットドロップ率}

パケットのドロップ率は、0\%だった。

\section{遅延の発生}
\label{sec:evaluation_latency}

実験に使用したアプリケーションでは、1ミリ秒間隔で1チャンネルあたり48サンプルずつ送信している。しかし、実験の結果から頻繁に1ミリ秒を超える間隔でパケットが送信され、受信したオーディオストリームを再生すると途中で音声が途切れる結果となった。
