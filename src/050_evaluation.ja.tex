\chapter{評価}
\label{chap:evaluation}

ベンチマークのシナリオを考えるだけ考えた。下記のいずれかをやるか、全部やるかについては今後の課題としたい。

\begin{itemize}
  \item マイク→オーディオインターフェイス→PC→AES67→ミキサー→AES67→PCまでの遅延計測
  \item PC→PCM音源→AES67→ミキサー→AES67→PCまでの遅延計測
  \item Dante Virtual Soundcard vs RAVENNA Virtual Audio Device vs 本研究で実装するAES67アプリケーション
\end{itemize}

また、伝送経路における遅延とジッタや、UDP通信におけるパケットロス率、アプリケーションを実行するコンピュータの環境についても評価に含めたい。特に、伝送経路中のL2スイッチが2台以上になった場合、どの程度遅延が増大するかを調査する。アプリケーションの実行環境については、つよいコンピュータ(第9世代Core i7)とよわいコンピュータ(Raspberry Pi 4/Zero)などの性能差や、LinuxカーネルにおけるTickパラメータの差異によってベンチマークのパフォーマンスに変化があるのか確認したい。

\section{評価環境}

\begin{table}[htb]
  \label{tab:evaluation_computer}
  \caption{コンピュータ評価環境}
  \centering
  \begin{tabular}{c|l} \hline
    OS & Windows 10 Pro バージョン 1909 \\ \hline
    CPU & Intel Core i7-9700K 3.60GHz \\ \hline
    RAM & 32GB \\ \hline
    SSD & Samsung SSD 970 EVO Plus 250GB \\ \hline
    NIC & StarTech ST1000SPEXI (Intel I210-AT) \\ \hline
  \end{tabular}
\end{table}

\begin{table}[htb]
  \label{tab:evaluation_network}
  \caption{ネットワーク評価環境}
  \centering
  \begin{tabular}{c|l} \hline
    ネットワークスイッチ & YAMAHA SWR2311P-10G \\ \hline
  \end{tabular}
\end{table}

\section{AES67伝送の結果}

\subsection{UDPパケット転送遅延}

最小0.01ms、最大0.05msの遅延が発生した。

\subsection{パケットドロップ率}

パケットのドロップ率は、0\%だった。

\section{RTPパケット送信間隔の適正値}

1ms間隔で48サンプルずつが正解なのか、10ms間隔で480サンプルずつが正解なのか
