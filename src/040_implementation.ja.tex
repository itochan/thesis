\chapter{実装}
\label{chap:implementation}

本章では、\ref{chap:design}章で示した設計を実装した際の詳細について述べる。

AES67の送受信アプリケーションは、Go言語で実装する。Go言語を採用した理由は、大きく3つの理由がある。

\begin{itemize}
  \item 現代的な静的型付け言語である
  \item 容易にシステムコールを呼び出すことが可能
  \item 異なるOSやアーキテクチャの上で動くバイナリを生成できる
\end{itemize}

Go言語でAES67の送受信アプリケーションを実装する。

\section{実装環境と対応プラットフォーム}

実装環境

\begin{itemize}
  \item{go version go1.13.6}
\end{itemize}

対応OS

\begin{itemize}
  \item Windows (\$GOOS=windows)
  \item macOS (\$GOOS=darwin)
  \item Linux (\$GOOS=linux)
\end{itemize}

対応アーキテクチャ

\begin{itemize}
  \item amd64 (\$GOARCH=amd64)
  \item armv7 (\$GOARCH=arm \$GOARM=7)
\end{itemize}

\section{SAPプロトコルのアナウンス}

SAPでは、RFC 2974\cite{rfc2974}で定められているマルチキャストアドレスとポートに機器の対応プロトコルやIPアドレス、ポート番号をアナウンスする。その内容にはRFC 4566\cite{rfc4566}で策定されたSession Description Protocol (SDP)を用いる。

本実装では、次のセッション告知を行なっている。なお、IPアドレスとホスト名に関しては一部マスキング処理を施している。

\begin{itembox}[l]{Session Description}
  \begin{verbatim}
    v=0
    o=- 4 0 IN IP4 203.178.XXX.YYY
    s=wifi-XXX-YYY.sfc.wide.ad.jp
    c=IN IP4 239.69.XXX.YYY/15
    t=0 0
    m=audio 5004 RTP/AVP 97
    c=IN IP4 239.69.XXX.YYY/15
    a=rtpmap:97 L24/48000/2
    a=sync-time:0
    a=framecount:48
    a=ptime:1
    a=recvonly
  \end{verbatim}
\end{itembox}

\begin{description}
  \item[v] SDPのプロトコルバージョン
  \item[o] オリジン: [ユーザネーム] [セッションID] [セッションバージョン] [ネットワークタイプ] [アドレスタイプ] [ユニキャストアドレス]
  \item[s] セッション名
  \item[c] コネクションデータ: [ネットワークタイプ] [アドレスタイプ] [コネクションアドレス]
  \item[t] タイミング: [スタートタイム] [ストップタイプ]
  \item[m] メディア説明: [メディア] [ポート] [プロトコル] [フォーマット]
  \item[a] 属性: [属性名]:[値]
\end{description}
