\chapter{実装}
\label{chap:implementation}

本章では、\ref{chap:design}章で示した設計を実装した際の詳細について述べる。

AES67の送受信アプリケーションは、Go言語で実装する。Go言語を採用した理由は、大きく3つの理由がある。

\begin{itemize}
  \item 現代的な静的型付け言語である
  \item 容易にシステムコールを呼び出すことが可能
  \item 異なるOSやアーキテクチャの上で動くバイナリを生成できる
\end{itemize}

なお、実装コードはGitHubのitochan/aes67-transmitterリポジトリ(\url{https://github.com/itochan/aes67-transmitter})で公開している。

\section{対応する環境とプラットフォーム}

実装するにあたり、Go言語の特色であるクロスコンパイル可能な特徴を生かし、複数のプラットフォームで動作可能とした。Raspberry PiのようなARMベースのプロセッサを搭載したコンピュータでも動作させるため、ARMv7アーキテクチャもサポートしている。

\subsection{実装環境}

\begin{itemize}
  \item{go version go1.13.6}
\end{itemize}

\subsection{対応OS}

\begin{itemize}
  \item Windows (\$GOOS=windows)
  \item macOS (\$GOOS=darwin)
  \item Linux (\$GOOS=linux)
\end{itemize}

\subsection{対応アーキテクチャ}

\begin{itemize}
  \item amd64 (\$GOARCH=amd64)
  \item armv7 (\$GOARCH=arm \$GOARM=7)
\end{itemize}

\section{SAPプロトコルのアナウンス}

SAPでは、RFC 2974\cite{rfc2974}で定められているマルチキャストアドレスとポートに機器の対応プロトコルやIPアドレス、ポート番号をアナウンスする。その内容にはRFC 4566\cite{rfc4566}で策定されたSession Description Protocol (SDP)を用いる。

本実装では、次のセッション告知を行なっている。なお、IPアドレスとホスト名に関しては一部マスキング処理を施している。

\begin{itembox}[l]{Session Description}
  \begin{verbatim}
    v=0
    o=- 4 0 IN IP4 203.178.XXX.YYY
    s=wifi-XXX-YYY.sfc.wide.ad.jp
    c=IN IP4 239.69.XXX.YYY/15
    t=0 0
    m=audio 5004 RTP/AVP 97
    c=IN IP4 239.69.XXX.YYY/15
    a=rtpmap:97 L24/48000/2
    a=sync-time:0
    a=framecount:48
    a=ptime:1
    a=recvonly
  \end{verbatim}
\end{itembox}

本来AES67では、PTP (Precision Time Protocol) v2経由でクロック同期する必要がある。今回の実装ではPTPv2によるクロック同期は行わないものとしているが、PTPv2の同期が行う場合は、上記マニフェストに以下のマニフェストを追加する。

\begin{itembox}[l]{Session Description with PTPv2}
  \begin{verbatim}
    a=mediaclk:direct=1810024580
    a=ts-refclk:ptp=IEEE1588-2008:39-A7-94-FF-FE-07-CB-D:0
  \end{verbatim}
\end{itembox}

\begin{description}
  \item[v] SDPのプロトコルバージョン
  \item[o] オリジン: [ユーザネーム] [セッションID] [セッションバージョン] [ネットワークタイプ] [アドレスタイプ] [ユニキャストアドレス]
  \item[s] セッション名
  \item[c] コネクションデータ: [ネットワークタイプ] [アドレスタイプ] [コネクションアドレス]
  \item[t] タイミング: [スタートタイム] [ストップタイプ]
  \item[m] メディア説明: [メディア] [ポート] [プロトコル] [フォーマット]
  \item[a] 属性: [属性名]:[値]
\end{description}

\section{RTP通信}

AES67のオーディオストリーム伝送には、RTPを用いる。RTP通信を行うにあたり、GoRTP\footnote{wernerd/GoRTP \url{https://github.com/wernerd/GoRTP}}ライブラリを利用した。後述する理由により、オリジナルのGoRTPからAES67に必要な改変を行ったものを利用している\footnote{itochan/GoRTP \url{https://github.com/itochan/GoRTP}}。

\subsection{IPパケットにおけるDSCP値の付加}

AES67では、DiffServベースのQoS(Quality of Service)が必須条件に含まれている。QoSは、パケットに優先順位を付け、ネットワークスイッチ内でどのパケットの処理を優先するか決めるものだ。DiffServは、IPヘッダにおけるDSフィールド中のDSCP値を参照し、優先度のクラスによりキューイングするものである\cite{yamaha_diffserv}。

GoRTPの実装では、UDPパケットのヘッダの値を変更することはできないため、AES67で必須条件のDSCP値を付加するための変更を行っている。

\subsubsection{QoSクラスとDiffServアソシエーション}

AES67で使用されているQoSクラスとDiffServアソシエーションを記す\cite{aes67-2018}。本実装では、以下のうちRTPパケットにおいてDSCP値34を付加して送信している。

\begin{description}
  \item[クラス名] Clock
  \item[トラフィックタイプ] IEEE 1588-2008 $Announce, Sync, Follow\_Up, Delay\_Req, Delay\_Resp, \\Pdelay\_Req, Pdelay\_Resp and Pdelay\_Resp\_Follow\_Up$ packets
  \item[デフォルトDiffServクラス(DSCP10進数値)] EF(46)
\end{description}

\hrulefill

\begin{description}
  \item[クラス名] Media
  \item[トラフィックタイプ] RTP and RTCP media stream data
  \item[デフォルトDiffServクラス(DSCP10進数値)] AF41 (34)
\end{description}

\hrulefill

\begin{description}
  \item[クラス名] Best effort
  \item[トラフィックタイプ] IEEE 1588-2008 signaling and management messages. Discovery and connection management messages.
  \item[デフォルトDiffServクラス(DSCP10進数値)] DF (0)
\end{description}

\subsection{RTPパケットのペイロード}

RTPパケットのペイロードは1パケット中で48サンプル送る場合において、PCM 24bit/48kHzであれば288バイトとなる\cite{aes67-pcap}。必要なRTPペイロードの計算式を式\ref{equ:rtp_payload}に示す。

% \begin{itembox}[l]{RTPパケットペイロードサイズの計算式}
  % \label{equ:rtp_payload}
  % \caption{RTPパケットペイロードサイズの計算式}
\begin{equation}
  \label{equ:rtp_payload}
  P = bsc
\end{equation}
\begin{itemize}
  \item $P$: total payload bytes [byte]
  \item $b$: bit depth [byte] (24bit: 3[bytes])
  \item $s$: sampling rate [kHz] (48kHz: 48[kHz])
  \item $c$: audio channels [channel] (Stereo channel: 2[channel])
\end{itemize}

送信機能では、PCMで記録されたオーディオファイルを288バイトずつ取り出し、RTPペイロードに載せたものを伝送する。

受信機能では、RTPペイロードからオーディオを取り出し、保存する。

\subsection{RTPパケットの送信間隔}

RTPパケットを送信する間隔は、安定したオーディオストリームの送信において重要だ。本実験で実装したアプリケーション同士で送受信を行えば、パケットの間隔を気にせずとも受信できると考えられる。しかし、受信側がAES67対応のオーディオ機器の場合において、同時に多くのパケットを受信するとバッファサイズを超え、処理能力の限界を超えることがある。一度に多くのオーディオストリームを受信し処理能力を超えると受信したオーディオストリームは切り捨てられ、音の断続的な途切れが発生する。

48kHzであれば1秒間に48サンプルを1000回送信することができれば、理想とする低遅延な伝送を行うことができる。実装したアプリケーションでは、1ミリ秒ごとに送信する実装となっている。
